%%%%%%%%%%%%%%%%%%%%%%%%%%%%%%%%%%%
%%   Design Document Template    %%
%%   For Physics Simulations     %%
%%      By Danny Hellstein       %%
%%      danielgoldstein.net      %%
%%%%%%%%%%%%%%%%%%%%%%%%%%%%%%%%%%%

\documentclass[]{scrartcl}
\usepackage{xcolor}
\usepackage[]{algorithm2e}
\usepackage{graphicx}
\title{Ising Model Design Document}
\author{Danny Hellstein}
\begin{document}
\definecolor{comment}{rgb}{0,.5,0}
\maketitle

\section{Introduction}
This simulation is an implementation of the Ising model. This is a model for ferromagnetic material. This code will investigate the phase transition and hysteresis of the Ising model.

\section{Mathematical Model}
The system consists of spins on a square lattice. Each spin $\sigma$ is points either up or down, represented as a $+1$ or $-1$. 

The Hamiltonian for this system is:
$$\mathcal{H} = -J\sum_{\langle ij\rangle}\sigma_i\sigma_j - H\sum_i \sigma_i$$
where $\displaystyle\sum_{\langle ij\rangle}$ denotes a sum over $i$ and its nearest neighbors (up, down, left, right) $j$. The interaction between neighboring spins is set by the interaction strength $J$. We can also apply an external magnetic field $H$.

\section{Computational Model}
		To simulate this system we will sample states at a temperature T. To generate these states we'll use the Metropolis Hasting algorithm we discussed in class. The procedure for a single Monte Carlo (MC) step is as follows
\begin{enumerate}
	\item Chose a random spin $i$.
	\item Calculate the energy $E_{inital} = \mathcal{H}$
	\item Flip the spin $\sigma_i = -\sigma_i$
	\item Calculate the new energy $E_{final} = \mathcal{H}$
	\item If the energy is now less $E_{inital}<E_{final}$ accept the change.
	\item If the energy is greater accept it with a probability $exp(-\frac{\Delta E}{k_b T})$.
	\item Otherwise flip the spin back
	\item Repeat this process a number of time equal to the number of spins in the system.\\
	Hint: you only need to calculate the change in energy for flipping a single spin. This should not involve the entire system. Only nearest neighbors.
	
\end{enumerate}


\section{Initial Conditions}
The system will be initialized with random initial conditions. Each spin has a $50\%$ chanse of being in the up $+1$, or down $-1$ state.

\section{Boundary Conditions}
We will employ periodic boundary conditions. Spins at the right wall will have spins at the left wall as their neighbors and vice versa. 


\section{Observables}
We will study the magnetization and the order parameter of the system. 
The magnetism is $m =\frac{\sum_i \sigma_i}{N}$ where $N$ is the total number of spins. Closely related is the order parameter $\phi = |m|$ that tells us how aligned the spins are.

\section{Experiments}
This code will run two experiments
\begin{enumerate}
\item Calculate $\phi$ as a function of T from on T =$[10,0]$. Taking 100 MC steps at each point to equilibrate, then averaging over 100 MC steps for the measurement. 
\item Calculate $m$ as a function of H for $T = 5$ and $T = 1$. Take H from 10 to -10 and back again for the same system. Taking 100 MC steps at each point to equilibrate, then averaging over 100 MC steps for the measurement. 
\end{enumerate}


\section{Simulation Interface}
This simulation will take in a command line argument for system size. It only accepts positive integers.

\pagebreak
\section{Code Structure}
Class Diagram
\begin{center}
	\includegraphics[width=\linewidth]{"Ising Classes"}
\end{center}
\pagebreak
Object Diagram
\begin{center}
	\includegraphics[width=\linewidth]{"Ising Objects"}
\end{center}
Runtime Diagram
\begin{center}
	\includegraphics[width=\linewidth]{"Ising Runtime"}
\end{center}

\end{document}
